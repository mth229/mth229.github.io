\documentclass[12pt]{article}
\usepackage[fleqn]{amsmath}     %puts eqns to left, not centered
\usepackage{graphicx}
\usepackage{hyperref}
\begin{html}
<style>
pre {font-size: 1.2em; background-color: #EEF0F5;}
ul li {list-style-image: url(http://www.math.csi.cuny.edu/static/images/julia.png);}  
</style>
\end{html}
\begin{document}

\section{Questions to be handed in on extrema:}

To get started, we load \texttt{Gadfly} so that we can make plots, and
load the \texttt{Roots} package for \texttt{D} and \texttt{fzero}:



\begin{verbatim}
using Gadfly            # ignore any warnings
using Roots         # for D and fzero
\end{verbatim}
\subsubsection{Quick background}

Read about this material here:
\href{http://mth229.github.io/extrema.html}{Maximization and
minimization with julia}.

For the impatient, \emph{extrema} is nothing more than a fancy word for
maximum \emph{or} minimum. In calculus, we have two concepts of these
\emph{relative} extrema and \emph{absolute} extrema. Let's focus for a
second on \emph{absolute} which are stated as:

\begin{quote}
A value $y=f(x)$ is an absolute maximum over an interval $[a,b]$ if
$y \geq f(x)$ for all $x$ in $[a,b]$. (An absolute minimum has
$y \leq f(x)$ instead.)
\end{quote}

There are two theorems which help identify extrema here. The first, due
to Bolzano, says that any continuous function on a \emph{closed}
interval will have an absolute maximum and minimum on that interval. The
second, due to Fermat, tells us where to look: these absolute maximums
and minimums can only occur at end points or critical points, then
evaluate to determine.

Bolzano and Fermat are historic figures. For us, we can plot a function
to visually see extrema. The value of Bolzano is the knowledge that yes,
plotting isn't a waste of time, as we are \emph{guaranteed} to see what
we look for. The value of Fermat is that if you want to get
\emph{precise} numeric answers, you have a means: identify the end
points and the critical points then compare.

The notes walk you through the task of finding among all rectangles with
perimeter 20 the one with maximum area. This is done quickly via:



\begin{verbatim}
A(b, h) = h * b             # base times height is area
## From P = 2b + 2h -> h = (P -2b)/2
h(b) = (20 - 2b)/2      # Relates perimeter, height, base
A(b) = A(b, h(b))       # substitution step: express Area in terms of b alone
x = fzero(D(A), [0, 10])    # critical point of A(b) on its domain
f(0), f(x), f(10)
\end{verbatim}
Here \texttt{x} is a critical point. Following Fermat, we checked the
value of the function at \texttt{x} along with the endpoints, $0$ and
$10$. However, a simple graph also illustrates that any maximum occurs
in between these endpoints (with the minimum occurring at both):



\begin{verbatim}
plot([A, D(A)], 0, 10)      # Notice zero of D(A) corresponds to maximum of A
\end{verbatim}
Notice what is done. The original problem had two variables (a base and
a height) and a fixed relationship between them (the perimeter is 20).
From this one variable can be deduced in terms of another leaving us a
continuous function (\texttt{A}) with extrema of interest (in this case
the maximum).

Notice how \texttt{julia} makes this easy: the substitution step of math
is done using \textbf{composition} of functions and leverages the fact
that \texttt{julia}, using \emph{multiple dispatch}, can give the
function \texttt{A} different meanings depending on the number (and
type) of its arguments: with one argument, the area is a function of the
base \texttt{b} with two arguments, the area is a function of the base
and the height.

To solve for the $x$-value corresponding to the extrema, we used
\texttt{fzero} with bracketing, as it is guaranteed to converge and it
is clear that the interval $[0,10]$ is a bracket for the derivative
function. We could also have identified a good initial guess for the
maximum from the graph, say 5, and just called \texttt{fzero} with this
initial guess, as \texttt{fzero(D(A), 5)}.

\subsubsection{Questions}

For the following questions (which were cribbed from various internet
sources), find the most precise answer you can, a graphical solution is
not enough, use one of the zero-finding methods to get as accurate
answer you can.

\begin{itemize}
\itemsep1pt\parskip0pt\parsep0pt
\item
  Ye olde post office
\end{itemize}

A box with a square base is taller than it is wide. In order to send the
box through the U.S. mail, the height of the box and the perimeter of
the base can sum to no more than 108 inches. Show how to compute the
maximum volume for such a box.

\begin{answer}
type: longtext
reminder: Maximum volume for box with 108 = h + 4w

rows: 6
cols: 60
\end{answer}

\begin{itemize}
\itemsep1pt\parskip0pt\parsep0pt
\item
  How big is that can?
\end{itemize}

A cylindrical can, \textbf{open on top}, is to hold 355 cubic
centimeters of liquid. Find the height and radius that minimizes the
amount of material needed to manufacture the can. (These are metric
units, so the answer will be in centimeters with 2.54cm=1in.) Illustrate
how you do this:

\begin{answer}
type: longtext
reminder: Minimize material in can
answer_text: \verb#SA(r,h)=2pi*r*h+pi*r^2;h(r)=355/(pi*r^2)# 
rows: 6
cols: 60
\end{answer}

Do these proportions match those you typically see for a 12 oz can?

\begin{answer}
type: radio
reminder: Do proportions match?
values: 1 | 2 | 3
labels: Yes, the height is about 2 times the diameter | No, the can has a square profile | No, the diameter is twice the height
answer: 3
\end{answer}

\begin{itemize}
\itemsep1pt\parskip0pt\parsep0pt
\item
  Cheap paper cups
\end{itemize}

A cone-shaped paper drinking cup is to hold 100 cubic centimeters of
water (about 4 ozs). Find the height and radius of the cup that will
require the least amount of paper. You may be interested to know that
the volume of such a cup is given by the volume of a cone formula:
$V = (1/3)\pi r^2 h$ and the area of the paper is given by
$A=\pi r \sqrt{r^2 + h^2}$. Show your work.

\begin{answer}
type: longtext
reminder: work to minimize paper used in conical drink cup
answer_text: \verb#h(r)=100/((1/3)*pi*r^2); A(r,h)=pi*r*sqrt(r^2+h^2); A(r)=A(r,h(r))#, r is around 4.0721 
rows: 6
cols: 60
\end{answer}

\begin{itemize}
\itemsep1pt\parskip0pt\parsep0pt
\item
  Inscription
\end{itemize}

A trapezoid is inscribed in a semicircle of radius $r=2$ so that one
side is along the diameter. Find the maximum possible area for the
trapezoid.

Draw a picture of a semicircle and a trapezoid. The trapezoid intersects
the circle at $(r,0)$, $(-r,0)$, $(r \cos(t), r\sin(t))$ and
$(-r\cos(t), r\sin(t))$ where $t$ is some angle in $[0, \pi/2]$. The
area of trapezoid is the height times the average of the two bases.

\begin{answer}
type: longtext
reminder: Commands to find maximum area of inscribed trapezoid
answer_text: \verb#r=2;A(t) = (2r+2r*cos(t))/2 * r*sin(t);fzero(D(A), 1) has 1.04; A=5.19# 
rows: 6
cols: 60
\end{answer}

\begin{itemize}
\itemsep1pt\parskip0pt\parsep0pt
\item
  Best size for a phone
\end{itemize}

A cell phone manufacturer wishes to make a rectangular phone with total
surface area of 12,000 $mm^2$ and maximal screen area. The screen is
surrounded by bezels with sizes of 8$mm$ on the long sides and 32$mm$ on
the short sides. (So the width of the screen would be the width of the
phone minus 2 time 8mm, and the height of the screen similarly
adjusted.)

What are the dimensions (width and height) that allow the maximum screen
area?

\begin{answer}
type: longtext
reminder: maximize printable area with 1/4-inch margins
answer_text: \verb+w(h)=12000/h; A(w,h)=(w-2*8)*(h-2*32); A(h)=A(w(h),h); h = fzero(D(A),100)->219..; w(h)=54... + 
rows: 6
cols: 60
\end{answer}

The dimensions for the iPhone 6 plus have height about twice the width.
Is that the case for your answer?

\begin{answer}
type: radio
reminder: 
values: 1 | 2 | 3 | 4
labels: No the height to width ratio is closer to 1 to 1 | Yes, the height to width ratio is basically 2 to 1 | No, the height to width ratio is closer to 3 to 1 | No the height to width ratio is closer to 4 to 1
answer: 4
\end{answer}

\begin{itemize}
\itemsep1pt\parskip0pt\parsep0pt
\item
  Will you be in the water?
\end{itemize}

The Statue of Liberty stands 92 meters high, including the pedestal
which is 46 meters high. How far from the base should you stand so that
your viewing angle, theta, is as large as possible?
\href{http://astro.temple.edu/~dhill001/maxmin/viewanglepic.gif}{figure}

\begin{answer}
type: longtext
reminder: Commands to maximize viewing angle
answer_text: \verb#Theta(x)=atan(92/x)-atan(46/x);fzero(D(Theta), 50)# is 65 
rows: 6
cols: 60
\end{answer}

\begin{itemize}
\itemsep1pt\parskip0pt\parsep0pt
\item
  Getting closer
\end{itemize}

Let $f(x) = \tan(x)$. Find the point on the graph of $f(x)$ that is
closest to the point $(\pi/4, 0)$. Show your work.

\begin{answer}
type: longtext
reminder: Minimize distance to graph
answer_text: \verb#f(x) = tan(x); d2(x)=(x-pi/4)^2+(f(x)-0)^2; fzero(D(d2), 0, pi/3)# is 0.35... 
rows: 6
cols: 60
\end{answer}

\end{document}

