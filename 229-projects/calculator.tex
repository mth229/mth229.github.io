\documentclass[12pt]{article}
\usepackage[fleqn]{amsmath}     %puts eqns to left, not centered
\usepackage{graphicx}
\usepackage{hyperref}
\begin{html}
<style>
pre {font-size: 1.2em; background-color: #EEF0F5;}
ul li {list-style-image: url(http://www.math.csi.cuny.edu/static/images/julia.png);}  
</style>
\end{html}
\begin{document}

\subsection{Questions to be handed in for project 1: Julia as a
calculator}

Read about this material here:
\href{http://mth229.github.io/calculator.html}{Julia as a calculator}.

\subsubsection{Expressions}

\begin{itemize}
\itemsep1pt\parskip0pt\parsep0pt
\item
  Compute the following value:
\end{itemize}

\[
(5/9)(-10 - 32)
\]

\begin{answer}
    type: numeric
    reminder: \( (5/9) * (-10 - 32) \)
    answer: [-23.334333333333337, -23.332333333333334]

\end{answer}

\begin{itemize}
\itemsep1pt\parskip0pt\parsep0pt
\item
  Compute the following value:
\end{itemize}

\[
9/5(100) + 32
\]

\begin{answer}
    type: numeric
    reminder: \( 9/5(100) + 32 \)
    answer: [211.999, 212.001]

\end{answer}

\begin{itemize}
\itemsep1pt\parskip0pt\parsep0pt
\item
  Compute the following value:
\end{itemize}

\[
-4.9\cdot 10^2 + 19.6\cdot 10 + 58.8 
\]

\begin{answer}
    type: numeric
    reminder: \( -4.9\cdot 10^2 + 19.6\cdot 10 + 58.8 \)
    answer: [-235.20100000000005, -235.19900000000004]

\end{answer}

\begin{itemize}
\itemsep1pt\parskip0pt\parsep0pt
\item
  Compute the following value:
\end{itemize}

\[
\frac{1 + 2\cdot 3}{4 + 5^6}
\]

\begin{answer}
    type: numeric
    reminder: \( \frac{1 + 2cdot 3}{4 + 5^6} \)
    answer: [-0.0005521146586473862, 0.0014478853413526138]

\end{answer}

\subsection{Math functions}

\begin{itemize}
\itemsep1pt\parskip0pt\parsep0pt
\item
  Compute the following value:
\end{itemize}

\[
\sqrt{0.25\cdot(1-0.25)/100}
\]

\begin{answer}
    type: numeric
    reminder: \( \sqrt{0.25\cdot(1-0.25)/100} \)
    answer: [0.04230127018922193, 0.044301270189221933]

\end{answer}

\begin{itemize}
\itemsep1pt\parskip0pt\parsep0pt
\item
  Compute the following value
\end{itemize}

\[
\cos^2(\pi/3)
\]

\begin{answer}
    type: numeric
    reminder: \( \cos^2(\pi/3) \)
    answer: [0.2490000000000001, 0.2510000000000001]

\end{answer}

\begin{itemize}
\itemsep1pt\parskip0pt\parsep0pt
\item
  Compute the following value:
\end{itemize}

\[
\sin^2(\pi/3)  \cdot \cos((\pi/6)^2)
\]

\begin{answer}
    type: numeric
    reminder: \( \sin^2(\pi/3)  \cdot \cos((\pi/6)^2) \)
    answer: [0.7209905957444216, 0.7229905957444216]

\end{answer}

\begin{itemize}
\itemsep1pt\parskip0pt\parsep0pt
\item
  Compute the following value:
\end{itemize}

\[
e^{(1/2)\cdot(3 - 2.3)^2}
\]

``

\begin{answer}
    type: numeric
    reminder: \( e^{(1/2)\cdot(3 - 2.3)^2} \)
    answer: [1.276621313204887, 1.2786213132048867]

\end{answer}

\begin{itemize}
\itemsep1pt\parskip0pt\parsep0pt
\item
  Compute the following value:
\end{itemize}

\[
1 + \frac{1}{2} + \frac{1}{2\cdot 3} + \frac{1}{2\cdot 3\cdot4} + \frac{1}{2\cdot 3\cdot4\cdot5}
\]

\begin{answer}
    type: numeric
    reminder: \(1 + \frac{1}{2} + \frac{1}{2\cdot 3} + \frac{1}{2\cdot 3\cdot4} + \frac{1}{2\cdot 3\cdot4\cdot5}\)
    answer: [1.715666666666667, 1.7176666666666667]
answer_text: \verb+1 + rac{1}{2} + rac{1}{2cdot 3} + rac{1}{2cdot 3cdot4} + rac{1}{2cdot 3cdot4cdot5}+ 
\end{answer}

\begin{itemize}
\itemsep1pt\parskip0pt\parsep0pt
\item
  Compute the following value:
\end{itemize}

\[
\frac{5}{\cos(57^\circ)}  + \frac{8}{\sin(57^\circ)}
\]

\begin{answer}
    type: numeric
    reminder: \( \frac{5}{\cos(57^\circ)}  + \frac{8}{\sin(57^\circ)} \)
    answer: [18.718298636570893, 18.720298636570895]

\end{answer}

\subsection{Precedence}

\begin{itemize}
\itemsep1pt\parskip0pt\parsep0pt
\item
  There are 5 operations in the following expression. Write a similar
  expression using 4 pairs of parentheses that evaluates to the same
  value:
\end{itemize}

\[
1 - 2 + 3 \cdot 4 ^ 5 / 6
\]

\begin{answer}
type: shorttext
reminder: \( 1 - 2 + 3 * 4 ^ 5 / 6 \)
answer: (1 - 2) + ((3 * (4 ^ 5)) / 6)
answer_text: \( (1 - 2) + ((3 * (4 ^ 5)) / 6) \) 
\end{answer}

\begin{itemize}
\item
  Which of these will also produce $1/(3\cdot4)$:
\item
  \texttt{1/3*4}
\item
  \texttt{1/3/4}
\item
  \texttt{1*3/4}
\end{itemize}

\begin{answer}
type: radio
reminder: \verb+1/(3cdot4)+
values: 1 | 2 | 3
labels: \verb+1/3*4+ | \verb+1/3/4+ | \verb+1*3/4+
answer: 2
\end{answer}

\subsection{Variable}

\begin{itemize}
\itemsep1pt\parskip0pt\parsep0pt
\item
  Let \texttt{x=4} and \texttt{y=7} compute
\end{itemize}

\[
x - \sin(x + y)/\cos(x - y)
\]

\begin{answer}
    type: numeric
    reminder: x - sin(x + y)/ cos(x - y)
    answer: [2.9889012265400097, 2.9909012265400094]

\end{answer}

\begin{itemize}
\itemsep1pt\parskip0pt\parsep0pt
\item
  For the polynomial
\end{itemize}

\[
y = ax^2 + bx + c
\]

Let $a=0.00014$, $b=0.61$, $c=649$, and $x=200$. What is $y$?

\begin{answer}
    type: numeric
    reminder: ax^2 + bx + c
    answer: [776.5999, 776.6001]

\end{answer}

\begin{itemize}
\itemsep1pt\parskip0pt\parsep0pt
\item
  If
\end{itemize}

\[
\frac{\sin(\theta_1)}{v_1} = \frac{sin(\theta_2)}{v_2}
\]

and $\theta_1 = \pi/5$, $\theta_2 = \pi/6$, and $v_1=2$, find $v_2$.

\begin{answer}
    type: numeric
    reminder: find \(v_2\)
    answer: [1.7012016167040798, 1.7014016167040797]

\end{answer}

\subsection{Some applications}

\begin{itemize}
\itemsep1pt\parskip0pt\parsep0pt
\item
  The period of simple pendulum depends on a gravitational constant
  $g=9.8$ and the pendulum length, $L$, in meters, according to the
  formula: $T=2\pi\sqrt{L/g}$.
\end{itemize}

A rope swing is timed to have a period of $6$ seconds. How long is the
length of the rope if the formula applies?

\begin{answer}
    type: numeric
    reminder: Find L
    answer: [8.936428397254193, 8.936628397254193]

\end{answer}

\begin{itemize}
\itemsep1pt\parskip0pt\parsep0pt
\item
  An object dropped from a building of height $h$ (in feet) will fall
  according to the laws of projectile motion:
\end{itemize}

\[
y(t) = h - 16t^2
\]

If $h=50$ find $y$ if $t=1.5$.

\begin{answer}
    type: numeric
    reminder: find y
    answer: [35.9999, 36.0001]

\end{answer}

\begin{itemize}
\itemsep1pt\parskip0pt\parsep0pt
\item
  Suppose $v = 2\cdot 10^8$ and $c = 3 \cdot 10^8$ compute
\end{itemize}

\[
\frac{1}{\sqrt{1 - v^2/c^2}}
\]

(Be careful, this expression from a theory of relativity is susceptible
to integer overflow on some computers!)

\begin{answer}
    type: numeric
    reminder: \( \frac{1}{\sqrt{1 - v^2/c^2}} \)
    answer: [1.3415407864998738, 1.3417407864998738]

\end{answer}

\subsection{Trig practice}

\begin{itemize}
\itemsep1pt\parskip0pt\parsep0pt
\item
  A triangle has sides $a=500$, $b=750$ and $c=901$. Is this a right
  triangle?
\end{itemize}

\begin{answer}
type: radio
reminder: is this a right triangle?
values: 1 | 2
labels: True | False
answer: 2
\end{answer}

\begin{itemize}
\itemsep1pt\parskip0pt\parsep0pt
\item
  The law of sines states for a triangle with angle $A$, $B$, and $C$
  and opposite sides labeled $a$, $b$, $c$ one has
\end{itemize}

\[ 
\sin(A)/a = \sin(B)/b = \sin(C)/c.
\]

If $A=115^\circ$, $a=123$, and $b=16$, find $B$ (in degrees).

\begin{answer}
    type: numeric
    reminder: find B
    answer: [0.12319560672431773, 0.12339560672431774]

\end{answer}

\begin{itemize}
\itemsep1pt\parskip0pt\parsep0pt
\item
  The law of cosines generalizes Pythagorean's theorem:
  $c^2 = a^2 +   b^2 - 2ab \cos(C)$. A triangle has sides $a=5$, $b=9$,
  and $c=8$. Find $C$.
\end{itemize}

\begin{answer}
    type: numeric
    reminder: find C
    answer: [1.0851782044993055, 1.0853782044993054]

\end{answer}

\subsection{Numbers}

\begin{itemize}
\itemsep1pt\parskip0pt\parsep0pt
\item
  \texttt{7e-10} is greater than \texttt{8e-9}?
\end{itemize}

\begin{answer}
type: radio
reminder: is 7e-10 greater 8e-9?
values: 1 | 2
labels: True | False
answer: 2
\end{answer}

\begin{itemize}
\itemsep1pt\parskip0pt\parsep0pt
\item
  The values \texttt{2}, \texttt{2.0}, \texttt{2 + 0im} and
  \texttt{2//1} are all the same and yet all different. How so?
\end{itemize}

\begin{answer}
type: longtext
reminder: The values 2, 2.0, 2 + 0im and 2//1 are all the same and yet all different. How so?
answer_text: different storage types 
rows: 3
cols: 60
\end{answer}

\begin{itemize}
\itemsep1pt\parskip0pt\parsep0pt
\item
  Compute
\end{itemize}

\[
2^{-1].
\]

(This isn't quite as easy as it looks.)

\begin{answer}
    type: numeric
    reminder: 
    answer: [0.49999999, 0.50000001]

\end{answer}

\end{document}

