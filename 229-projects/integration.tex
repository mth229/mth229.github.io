\documentclass[12pt]{article}
\usepackage[fleqn]{amsmath}     %puts eqns to left, not centered
\usepackage{graphicx}
\usepackage{hyperref}
\begin{html}
<style>
pre {font-size: 1.2em; background-color: #EEF0F5;}
ul li {list-style-image: url(http://www.math.csi.cuny.edu/static/images/julia.png);}  
</style>
\end{html}
\begin{document}
\section{Questions to be handed in on integration:}\newline
To get started, we load the \texttt{Gadfly} backend for \texttt{Plots}, so that we can make plots; load the \texttt{Roots} package for its \texttt{D} function; and the \texttt{SymPy} package:\begin{verbatim}
using Plots
gadfly()
using Roots			# for D and fzero
using SymPy
\end{verbatim}
\subsubsection{Quick background}\newline
Read more about this material here: \href{http://mth229.github.io/integration.html}{integration}.\newline
For the impatient, in many cases, the task of evaluating a definite integral is made easy by the fundamental theorem of calculus which says that for a continuous function $f$ the following holds for any antiderivate, $F$, of $f$:
$$
\int_a^b f(x) dx = F(b) - F(a).
$$
\newline
That is the definite integral is found by evaluating a related function at the endpoints, $a$ and $b$.\newline
The \texttt{SymPy} package can compute many antiderivatives using a version of the \href{http://en.wikipedia.org/wiki/Risch\_algorithm}{Risch algorithm} that works for \href{http://en.wikipedia.org/wiki/Elementary\_function}{elementary functions}. \texttt{SymPy}'s \texttt{integrate} function can be used to find an indefinite integral:\begin{verbatim}
f(x) = x^2
integrate(f)
\end{verbatim}
$$\frac{x^{3}}{3}$$\newline
Or a definite integral:\begin{verbatim}
integrate(f, 0, 1)		# returns a "symbolic" number
\end{verbatim}
$$\frac{1}{3}$$\newline
However, this only works \textit{if} there is a known antiderivative $F(x)$ $-$ which is not always the case. If not, what to do?\newline
In this case, we can appeal to the definition of the definite integral. For continuous, non-negative $f(x)$, the definite integral is the area under the graph of $f$ over the interval $[a,b]$. For possibly negative functions, the indefinite integral is found by the \textit{signed} area under $f$.  This area can be directly \textit{approximated} using Riemann sums, or some other approximation scheme.\newline
The Riemann approximation can help. The following pattern will compute a Riemann sum using right-hand endpoints:\begin{verbatim}
f(x) = x^2
a, b, n = 0, 1, 5		# 5 partitions of [0,1] requested
delta = (b - a)/n		# size of partition
xs = a + (1:n) * delta	
fxs = [f(x) for x in xs]
sum(fxs * delta)		# a new function `sum` to add up values in a container
\end{verbatim}
\begin{verbatim}
0.44000000000000006\end{verbatim}
\newline
That value isn't very close to $1/3$. But we only took $n=5$ rectangles $-$ clearly there will be some error. Bigger $n$s mean better approximations:\begin{verbatim}
f(x) = x^2
a, b, n = 0, 1, 50_000		# 50,000 partitions of [0,1] requested
delta = (b - a)/n		
xs = a + (1:n) * delta	
fxs = [f(x) for x in xs]
sum(fxs * delta)
\end{verbatim}
\begin{verbatim}
0.3333433334\end{verbatim}
\newline
Note that only the first two lines needed changing to adjust to a new problem. As the pattern is similar, it is fairly easy to wrap the computations in a function for convenience. We borrow this more elaborate one from the notes that works for some other methods beside the default right-Riemann sum:\begin{verbatim}
function riemann(f::Function, a::Real, b::Real, n::Int; method="right")
  if method == "right"
     meth(f,l,r) = f(r) * (r-l)
  elseif method == "left"
     meth(f,l,r) = f(l) * (r-l)
  elseif method == "trapezoid"
     meth(f,l,r) = (1/2) * (f(l) + f(r)) * (r-l)
  elseif method == "simpsons"
     meth(f,l,r) = (1/6) * (f(l) + 4*(f((l+r)/2)) + f(r)) * (r-l)
  end

  xs = a + (0:n) * (b-a)/n
  as = [meth(f, l, r) for (l,r) in zip(xs[1:end-1], xs[2:end])]
  sum(as)
end
\end{verbatim}
\begin{verbatim}
riemann (generic function with 1 method)\end{verbatim}
\newline
The Riemann sum is very slow to converge here. There are faster algorithms both mathematically and computationally. We will discuss two: the trapezoid rule, which replaces rectangles with trapezoids; and Simpson's rule which is a quadratic approximation.\begin{verbatim}
f(x) = x^2
riemann(f, 0, 1, 1000, method="trapezoid"), riemann(f, 0, 1, 1000, method="simpsons")
\end{verbatim}
\begin{verbatim}
(0.33333350000000006,0.3333333333333337)\end{verbatim}
\newline
Base \texttt{julia} provides the \texttt{quadgk} function which uses a different approach altogether. It is used quite easily:\begin{verbatim}
f(x) = x^2
ans, err = quadgk(f, 0, 1)
\end{verbatim}
\begin{verbatim}
(0.3333333333333333,5.551115123125783e-17)\end{verbatim}
\newline
The \texttt{quadgk} function returns two values, an answer and an estimated maximum possible error.  The ans is the first number, clearly it is $1/3$, and the estimated maximum error is the second. In this case it is small ($10^{-17}$) $-$ basically 0.\subsubsection{Questions}\begin{itemize}\item Let $g(x) = x^4 + 10x^2 - 60x + 71$. Find the integral $\int_0^1   g(x) dx$ by hand by finding an antiderivative and then using the fundamental theorem of calculus.\end{itemize}
\\begin{answer}
type: longtext
reminder: Commands to do FTC
answer_text: \verb# G(x) = x^5/5 + 10x^3/3 -60x^2/2+71x; G(1)-G(0)=44.53# (or \verb#integrate(g, 0, 1)=668/15#) 
rows: 3
cols: 60
\\end{answer}
\begin{itemize}\item For $f(x) = x/\sqrt{g(x)}$ (for $g(x)$ from the last problem) estimate the following using 1000 Riemann sums:\end{itemize}
$$
\int_0^1 f(x) dx
$$

\\begin{answer}
    type: numeric
    reminder: riemann sum, n=1000
    answer: [0.08578252384025933, 0.08578254384025932]
    answer_text: [0.086, 0.086] 
\\end{answer}
\begin{itemize}\item Let $f(x) = \sin(\pi x^2)$. Estimate $\int_0^1 f(x) dx$ using 20 right-Riemann sums\end{itemize}
\\begin{answer}
    type: numeric
    reminder: riemann sum, n=20
    answer: [0.5035434308446651, 0.5035434508446652]
    answer_text: [0.504, 0.504] 
\\end{answer}
\begin{itemize}\item For the same $f(x)$, compare your estimate with 20 Riemann sums to   that with 20,000 Riemann sums. How many digits after the decimal   point do they agree?\end{itemize}
\\begin{answer}
type: radio
reminder: Where do they differ?
values: 3 | 6 | 4 | 1 | 2 | 5
labels: They differ at the fourth place after the decimal point | They differ at the fifth place after the decimal point | They differ at the first place after the decimal point | They differ at the third place after the decimal point | They differ at the sixth place after the decimal point | They differ at the second place after the decimal point
answer: 4

\\end{answer}
\subsubsection{Left Riemann}\newline
The left Riemann sum uses left-hand endpoints, not right-hand ones. \begin{itemize}\item For $f(x) = e^{x}$ use the left Riemann sum with $n=10,000$ to estimate $\int_0^1 f(x) dx$.\end{itemize}
\\begin{answer}
    type: numeric
    reminder: left-hand intervals
    answer: [1.718195905799352, 1.718195925799352]
    answer_text: [1.718, 1.718] 
\\end{answer}
\begin{itemize}\item The left and right Riemann sums for an increasing function are also   lower and upper bounds for the answer. Find the difference between   the left and right Riemann sum for $\int_0^1 e^x dx$ when   $n=10,000$. (Use your last answer.) What is the approximate value   $1/100$, $1/1000$, $1/10000$, or $1/100000$?\end{itemize}
\\begin{answer}
type: radio
reminder: Approximate difference for 10,000 steps
values: 1 | 2 | 3 | 4
labels: 1/100 | 1/10000 | 1/1000 | 1/100000
answer: 2

\\end{answer}
\subsubsection{Trapezoid, Simpson's}\begin{itemize}\item The answer to $\int_0^1 e^x dx$ is simply $e^1 - e^0$ =   $e-1$. Compare the error (in absolute value) of the trapezoid method when $n=10,000$.\end{itemize}
\\begin{answer}
type: radio
reminder: Size of error
values: 4 | 1 | 5 | 3 | 2 | 6 | 7
labels: The error is about 1e-10 | The error is about 1e-8 | The error is about 1e-6 | The error is about 1e-7 | The error is about 1e-9 | The error is about 1e-12 | The error is about 1e-11
answer: 5

\\end{answer}
\begin{itemize}\item The answer to $\int_0^1 e^x dx$ is simply $e^1 - e^0$ =   $e-1$. Compare the error of the Simpson's method when $n=10,000$.\end{itemize}
\\begin{answer}
type: radio
reminder: Size of error, simpsons
values: 1 | 6 | 4 | 5 | 3 | 7 | 2
labels: The error is about 1e-10 | The error is about 1e-13 | The error is about 1e-11 | The error is about 1e-15 | The error is about 1e-16 | The error is about 1e-14 | The error is about 1e-12
answer: 2

\\end{answer}
\newline
(The error for Riemann sums is "like" $1/n$, the error for trapezoid sums is like $1/n^2$, and for Simpson's rule the error is like $1/n^4$.)\subsection{quadgk}\begin{itemize}\item Use \texttt{quadgk} to find $\int_{-3}^{3} (1 + x^2)^{-1} dx$. What is the answer? What is the estimated maximum error?\end{itemize}\newline
The answer is:
\\begin{answer}
    type: numeric
    reminder: area under f
    answer: [2.4980915347965102, 2.49809155479651]
    answer_text: [2.498, 2.498] 
\\end{answer}
\newline
The error is about
\\begin{answer}
type: radio
reminder: Size of error, quadgk
values: 7 | 5 | 3 | 4 | 6 | 1 | 2
labels: The error is about 1e-10 | The error is about 1e-6 | The error is about 1e-9 | The error is about 1e-8 | The error is about 1e-12 | The error is about 1e-7 | The error is about 1e-11
answer: 4

\\end{answer}
\begin{itemize}\item Use \texttt{quadgk} to find the integral of $e^{-|x|}$ over $[-1,1]$.\end{itemize}
\\begin{answer}
    type: numeric
    reminder: area under f
    answer: [1.2642411076571154, 1.2642411276571153]
    answer_text: [1.264, 1.264] 
\\end{answer}
\begin{itemize}\item The integral of $\sqrt{1-x^4}$ over $[-1,1]$ can not be found with the Fundamental Theorem of Calculus using an elementary function for an antiderivative. What is the \textit{approximate} value?\end{itemize}
\\begin{answer}
    type: numeric
    reminder: area under f
    answer: [1.7480383606399779, 1.7480383806399777]
    answer_text: [1.748, 1.748] 
\\end{answer}
\begin{itemize}\item The integral of $f(x) = \log(log(x))$ over $[e,e^2]$ can not be   found with the Fundamental Theorem of Calculus using an elementary   function for an antiderivative. What is the \textit{approximate} value?\end{itemize}
\\begin{answer}
    type: numeric
    reminder: area under f
    answer: [2.0625868523270956, 2.0625868723270955]
    answer_text: [2.063, 2.063] 
\\end{answer}
\newline
The graph of $f(x)$ over the interval $[e, e^2]$ makes clear that the triangle formed by the line connecting $(e, f(e))$ and $(e^2, f(e^2))$, the $x$ axis, and the line $x=f(e^2)$ will form a lower bound for the area under $f$. What is the error in this approximation? (Where error = answer $-$ approximation.)
\\begin{answer}
    type: numeric
    reminder: area under f
    answer: [0.4437198540224312, 0.44391985402243117]
    answer_text: [0.444, 0.444] 
\\end{answer}
\begin{itemize}\item A formula to compute the length of a the graph of the function $f(x)$ from $a$ to $b$ is given by the formula:\end{itemize}
$$
\int_a^b \sqrt{1 + f'(x)^2} dx
$$
\newline
Use this formula when $f(x) = \sin(x)$ and the interval is $[0,\pi]$. What is the answer?
\\begin{answer}
    type: numeric
    reminder: 
    answer: [3.819197789027713, 3.8211977890277127]
    answer_text: [3.819, 3.821] 
\\end{answer}
\newline
Repeat, when the function is $f(x) = x^x$ over $(0, 3)$:
\\begin{answer}
    type: numeric
    reminder: 
    answer: [27.483739678027934, 27.485739678027937]
    answer_text: [27.484, 27.486] 
\\end{answer}
\begin{itemize}\item Compute the area between the intersection points of the two curves $f(x) = x$ and $g(x) = x^3$ by taking the difference between two definite integrals.\end{itemize}
\\begin{answer}
    type: numeric
    reminder: 
    answer: [0.24900000000000005, 0.25100000000000006]
    answer_text: [0.249, 0.251] 
\\end{answer}
\subsection{Applications}\newline
We discuss an application of the integral to finding the volumes $-$ not just areas.\newline
A \textit{solid of revolution} is a figure with rotational symmetry around some axis, such as a soda can, a snow cone, a red solo cup, and other common objects. A formula for the volume of an object with rotational symmetry can be written in terms of an integral based on a function, $r(h)$, which specifies the radius for various values of $h$.\begin{quotation}\newline
If the radius as a function of height is given by $r(h)$, the the volume is $\int_a^b \pi r(h)^2 dh$.\end{quotation}\newline
So for example, a baseball has a overall diameter of $2\cdot 37$mm, but if we place the center at the origin, its rotational radious is given by $r(h) = (37^2 - h^2)^{1/2}$ for $-37 \leq h \leq 37$. The volume in mm$^3$ is given by:\begin{verbatim}
r(h) = (37^2 - h^2)^(1/2)
v(h) = pi * r(h)^2
quadgk(v, -37, 37)
\end{verbatim}
\begin{verbatim}
(212174.79024304505,2.9103830456733704e-11)\end{verbatim}
\subsubsection{Glass half full}\begin{itemize}\item A glass is formed as a volume of revolution with radius as a   function of height given the equation $r(h) = 2 + (h/20)^4$. The   volume as a function of height $b$ is given by $V(b) = \int_0^b \pi   r(h)^2 dh$. Find $V(25)$. Show your work.\end{itemize}
\\begin{answer}
type: longtext
reminder: Volume of glass
answer_text: \verb#R(h) = 2 + (h/20)^4;V(b) = quadgk(x->pi*R(x)^2, 0, b)[1];V(25)# 519... 
rows: 3
cols: 60
\\end{answer}
\begin{itemize}\item Find a value of $b$ so that $V(b) = 455$.\end{itemize}
\\begin{answer}
type: longtext
reminder: b so that V(b) = 455
answer_text: \verb#fzero(b->V(b)-455, 25)# 23.85 
rows: 3
cols: 60
\\end{answer}
\begin{itemize}\item Now find a value of $b$ for which $V(b) = 455/2$. This height will   have half the volume as the height just found. Compare the two   values. Is the ratio of smallest to largest 1/2, more than 1/2 or   less?\end{itemize}
\\begin{answer}
type: longtext
reminder: b so that V(b) = 455/2
answer_text: \verb#fzero(b->V(b)-455/2, 25/2)# 16.48 
rows: 3
cols: 60
\\end{answer}

\end{document}
