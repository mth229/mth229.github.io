\documentclass[12pt]{article}
\usepackage[fleqn]{amsmath}     %puts eqns to left, not centered
\usepackage{graphicx}
\usepackage{hyperref}
\begin{html}
<style>
pre {font-size: 1.2em; background-color: #EEF0F5;}
ul li {list-style-image: url(http://www.math.csi.cuny.edu/static/images/julia.png);}  
</style>
\end{html}
\begin{document}

\section{Questions to be handed in for project 2: Functions in Julia}

Read about this \href{http://mth229.github.io/functions.html}{here}.

For the impatient, it is easy with \texttt{Julia} to create basic
mathematical functions, as the notation is identical:



\begin{verbatim}
f(x) = x^2 - 2
f(2)
\end{verbatim}
As with typical mathematical notation the function is called by passing
a value to it with parentheses.

That covers most of the uses within this project, but there is more...

\begin{enumerate}
\def\labelenumi{\arabic{enumi})}
\itemsep1pt\parskip0pt\parsep0pt
\item
  Julia functions can have more than one variable:
\end{enumerate}



\begin{verbatim}
rectarea(b, h) = b * h      # area of rectangle
\end{verbatim}
\begin{enumerate}
\def\labelenumi{\arabic{enumi})}
\setcounter{enumi}{1}
\itemsep1pt\parskip0pt\parsep0pt
\item
  Julia functions with different signatures, can have the same name!
\end{enumerate}



\begin{verbatim}
rectarea(b) = rectarea(b,b) # area of square using area of rectangle
\end{verbatim}
\begin{enumerate}
\def\labelenumi{\arabic{enumi})}
\setcounter{enumi}{2}
\itemsep1pt\parskip0pt\parsep0pt
\item
  Multiline functions are defined with the \texttt{function} keyword
\end{enumerate}



\begin{verbatim}
function multiline_function(x)
     top = x^2 - 2x
     bottom = x^4 - 4x
     top/bottom     # last line is return value
end             # don't forget the end
\end{verbatim}
\begin{enumerate}
\def\labelenumi{\arabic{enumi})}
\setcounter{enumi}{3}
\itemsep1pt\parskip0pt\parsep0pt
\item
  Functions can have keyword arguments (like parameters). These have
  default values:
\end{enumerate}



\begin{verbatim}
mxplusb(x; m=0, b=1) = m*x + b
mxplusb(10, m=3, b=4)       # arguments are named
\end{verbatim}
\begin{enumerate}
\def\labelenumi{\arabic{enumi})}
\setcounter{enumi}{4}
\itemsep1pt\parskip0pt\parsep0pt
\item
  Functions can be used as arguments to other functions:
\end{enumerate}



\begin{verbatim}
using Gadfly            # for plotting
plot(sin, 0, 2pi)       # pass function for first argument plot(f,a,b)
\end{verbatim}
\begin{enumerate}
\def\labelenumi{\arabic{enumi})}
\setcounter{enumi}{5}
\itemsep1pt\parskip0pt\parsep0pt
\item
  Functions can be \emph{anonymous}. These are defined via the pattern:
  "\texttt{args -\textgreater{} body}". Look for the arrow
  \texttt{-\textgreater{}}:
\end{enumerate}



\begin{verbatim}
plot(x -> sin(x)*cos(2x), 0, 2pi)
\end{verbatim}
\begin{enumerate}
\def\labelenumi{\arabic{enumi})}
\setcounter{enumi}{6}
\itemsep1pt\parskip0pt\parsep0pt
\item
  Functions can be \emph{returned} by other functions (see the "arrow"):
\end{enumerate}



\begin{verbatim}
secant_line(f, a, b) = x -> f(a) + (f(b) - f(a))/(b-a)*(x-a)
\end{verbatim}
\begin{center}\rule{3in}{0.4pt}\end{center}

\paragraph{Controlling what gets returned by \texttt{IJulia}}

When \texttt{IJulia} evaluates a cell, it will only display the last
command executed (without explicitly printing). One useful trick to
output more than one value at a time is to place them into a container.
Grouping values within parentheses will produce a "tuple":



\begin{verbatim}
(1, 2, "a")
\end{verbatim}
Tuples will be implicitly constructed if the parentheses are left off.
So separating values by commas can be used to have multiple values
printed in a cell.

If the values are all of the same type, the preferred container would be
made with \texttt{{[}{]}}, as in:



\begin{verbatim}
[1, 2]
\end{verbatim}
If the comma is left out, the results are combined "horizontally", which
can be a space saver in some cases:



\begin{verbatim}
[1 2 3 4 5]
\end{verbatim}
\begin{center}\rule{3in}{0.4pt}\end{center}

\subsubsection{Questions}

\begin{itemize}
\itemsep1pt\parskip0pt\parsep0pt
\item
  Write a function that describes a line with slope $2$ going through
  the point $(1,3)$. What is the value of $f(10)$?
\end{itemize}

The function is:

\begin{answer}
type: shorttext
reminder: function m=2, p = (1,3)
answer: f(x) = 3+2*(x-1)
answer_text: f(x) = 3+2*(x-1) 
\end{answer}

The value of $f(10)$ is:

\begin{answer}
    type: numeric
    reminder: f(10)
    answer: [20.9999, 21.0001]

\end{answer}

\begin{itemize}
\itemsep1pt\parskip0pt\parsep0pt
\item
  Write a function to convert Celsius to Fahrenheit $F=9/5 C +   32$.
  Use it to find the Fahrenheit value when $C=56.7$ and when $C=-89.2$.
  (The record high and low temperatures.)
\end{itemize}

The function is

\begin{answer}
type: shorttext
reminder: F=9/5 C + 32
answer: f(x) = 9/5*x+32
answer_text: f(x) =  9/5*x+32 
\end{answer}

The value at $C=56.7$ is

\begin{answer}
    type: numeric
    reminder: Fahrenheit at 56.7C
    answer: [134.0599, 134.0601]

\end{answer}

The value at $C=-89.2$ is:

\begin{answer}
    type: numeric
    reminder: Fahrenheit at -89.2C
    answer: [-128.5601, -128.5599]

\end{answer}

\begin{itemize}
\itemsep1pt\parskip0pt\parsep0pt
\item
  Write a function that computes
\end{itemize}

\[
f(x) = 4x^2 - 3x  - 7 - \frac{1}{x}
\]

Use it to find the values of $f(1)$, and $f(3)$.

The function is defined by:

\begin{answer}
type: shorttext
reminder: \( 4x^2 - 3x  - 7 - \frac{1}{x} \)
answer: f(x) = 4x^2 - 3x  - 7 - 1/x

\end{answer}

The value $f(1)$ is

\begin{answer}
    type: numeric
    reminder: 
    answer: [-7.0001, -6.9999]

\end{answer}

The value $f(3)$ is

\begin{answer}
    type: numeric
    reminder: 
    answer: [19.666566666666668, 19.666766666666668]

\end{answer}

\begin{itemize}
\itemsep1pt\parskip0pt\parsep0pt
\item
  Write a function that computes:
\end{itemize}

\[
f(t) = A\sin(Bt - C) + D
\]

where $A=3.1$, $B=2\pi/365$, $C=1.35$, and $D=12.12$.

This function models the amount of daylight in Boston when $t$ records
the day of the year. How much daylight is there for $t=1$, $t=365/2$,
$t = 35$?

The function is

\begin{answer}
type: shorttext
reminder: Asin(Bt - C) + D
answer: f(t) = 3.1 * sin(2pi/365*t - 1.35) + 12.12
answer_text: \( f(t) = 3.1\sin(2pi/365*t - 1.35) + 12.12 \) 
\end{answer}

The value at $t=1$ is

\begin{answer}
    type: numeric
    reminder: value at 1 (hours of daylight 1/1
    answer: [9.107292243291802, 9.107492243291802]

\end{answer}

The value at $t=365/2$ is:

\begin{answer}
    type: numeric
    reminder: value at 365/2 (hours of daylight at summer
    answer: [15.144642409262643, 15.144842409262642]

\end{answer}

The value at $t=35$ is:

\begin{answer}
    type: numeric
    reminder: value at 35
    answer: [9.095157590737356, 9.095357590737356]

\end{answer}

\begin{itemize}
\itemsep1pt\parskip0pt\parsep0pt
\item
  Person A starts at the origin and moves west at 60 MPH. Person B
  starts 200 miles north of the origin and moves south at 70 MPH. Write
  a function that computes the distance between the two people as a
  function of $t$ in minutes.
\end{itemize}

(The $(x,y)$ position of person A is $(60\cdot t/60), 0)$ and the
$(x,y)$ position of person B is $(0, 200 - 70 \cdot t/60)$. Use the
distance formula to write a function.)

The distance at $t=0$ is:

\begin{answer}
    type: numeric
    reminder: value at t=0
    answer: [199.9999, 200.0001]

\end{answer}

The distance at $t=30$ is:

\begin{answer}
    type: numeric
    reminder: value at t=30
    answer: [167.70499831248424, 167.70519831248424]

\end{answer}

The distance at $t=120$ is:

\begin{answer}
    type: numeric
    reminder: value at t=120
    answer: [134.16397864998737, 134.16417864998738]

\end{answer}

\subsection{Cases}

The ternary operator \texttt{predicate ? expression1 : expression2} has
three pieces: a predicate question, such as \texttt{x \textless{} 10}
and two expressions, the first is evaluated if the predicate is
\texttt{true} and the second if the predicate is \texttt{false}. They
are useful to write functions that are defined by cases. (They are a
light-weight form of the traditional \texttt{if-then-else} construct.)

\begin{itemize}
\itemsep1pt\parskip0pt\parsep0pt
\item
  Write a function to express the following: If a person buys up to 100
  units the cost per unit is 5 dollars, for every additional unit beyond
  100 the cost is 4 dollars. The function should return the total cost
  to buy $x$ units. (Use the ternary operator with
  \texttt{x \textless{}=   100} as the condition.
\end{itemize}

\begin{answer}
type: longtext
reminder: Total cost for x units
answer_text: \verb#f(x) = x <= 100 ? 5.0 * x : (5.0*100) + 4*(x-100) # 
rows: 3
cols: 60
\end{answer}

\begin{itemize}
\itemsep1pt\parskip0pt\parsep0pt
\item
  Use the ternary operator to write a function $f(x)$ which takes a
  value of $x$ when $x$ is less than 10 and is otherwise a constant
  value of 10.
\end{itemize}

\begin{answer}
type: shorttext
reminder: Function which is x for x < 10 and 10 otherwise
answer: f(x) = x < 10 ? x : 10

\end{answer}

\begin{itemize}
\itemsep1pt\parskip0pt\parsep0pt
\item
  Define a function $Sin(x) = \sin(x)$ if $-\pi/2 < x < \pi/2$ and
  \texttt{NaN} otherwise.
\end{itemize}

\begin{answer}
type: shorttext
reminder: restrict sin(x) to -pi/2 to pi/2
answer: Sin(x) = -pi/2 < x < pi/2 ? sin(x) : NaN

\end{answer}

\begin{itemize}
\itemsep1pt\parskip0pt\parsep0pt
\item
  Define a function
\end{itemize}

\[
signal(x) = 0 \text{ if } x\leq 0,
\] \[
\sin^2(\pi /2 x) \text{ if } 0<x<1, \text{ and }
\] \[
1 \text{ if } x\geq 1 
\]

Compare the difference between \texttt{signal(0.999)} and
\texttt{signal(1.001)}.

The function is

\begin{answer}
type: shorttext
reminder: signal function
answer: signal(x) = (x <= 0) ? 0.0 : ((x <= 1) ? sin(pi/2*x)^2 : 1.0)
answer_text: signal(x) = (x <= 0) ? 0.0 : ((x <= 1) ? sin(pi/2*x)^2 : 1.0) 
\end{answer}

The value at $0.999$ is:

\begin{answer}
    type: numeric
    reminder: f(0.9999)
    answer: [0.9998999753259891, 1.0000999753259892]

\end{answer}

The value at $1.0001$ is:

\begin{answer}
    type: numeric
    reminder: f(1.0001)
    answer: [0.9999, 1.0001]

\end{answer}

\subsection{Composition}

Composition of functions is a useful means to break complicated problems
into easier to solve one. The math notation is typically $f(g(x))$ and
in \texttt{julia} this is no different. When thinking about the
operation of composition, the notation $f \circ g$ is used. For that,
there isn't any built-in \texttt{julia} notation.

\begin{itemize}
\itemsep1pt\parskip0pt\parsep0pt
\item
  For the function $h(x) = ((x+1)/(x-1))^{2/3}$ write this as the
  composition of two functions $f(x)$ and $g(x)$. Use these to evaluate
  $h(3)$. Show your work and the answer.
\end{itemize}

\begin{answer}
type: longtext
reminder: write h(x) = ((x+1)/(x-1))^(2/3) as a composition, f(g(x)) and compute it at 10

rows: 3
cols: 60
\end{answer}

\begin{itemize}
\itemsep1pt\parskip0pt\parsep0pt
\item
  Write the function $h(x) = (\cos(12x))^3$ as the composition of two
  functions $f(x)$ and $g(x)$ and use these to evaluate $h(2)$. Show
  your work and the answer.
\end{itemize}

\begin{answer}
type: longtext
reminder: write h(x) =  (cos(12x))^3 as a composition, f(g(x)) and compute it at 2

rows: 3
cols: 60
\end{answer}

\subsection{Parameters}

\begin{itemize}
\itemsep1pt\parskip0pt\parsep0pt
\item
  The formula for a
  \href{http://en.wikipedia.org/wiki/Catenary}{catenary} has a parameter
  $a$:
\end{itemize}

\[
y = a \cosh(x/a)
\]

($\cosh(x)$ is the hyperbolic cosine, defined by
$(1/2) \cdot (e^x + e^{-x})$.

Write a function, \texttt{c(x;a=1)}, with \texttt{a} as a parameter
defaulting to $1$. Compute \texttt{c(1)}, \texttt{c(1,a=2)}, and
\texttt{c(1, a=1/2)}.

The function is:

\begin{answer}
type: shorttext
reminder: \( y = a \cosh(x/a) \)
answer: c(x; a=1) = a * cosh(x/a)
answer_text: c(x; a=1) = a * cosh(x/a) 
\end{answer}

The value of \texttt{c(1)}:

\begin{answer}
    type: numeric
    reminder: c(1)
    answer: [1.5429806348152437, 1.5431806348152437]

\end{answer}

The value of \texttt{c(1, a=2)}:

\begin{answer}
    type: numeric
    reminder: c(1, a=2)
    answer: [2.255151930412761, 2.2553519304127616]

\end{answer}

The value of \texttt{c(1, a=1/2)}:

\begin{answer}
    type: numeric
    reminder: c(1, a=1/2)
    answer: [1.8809978455418157, 1.8811978455418157]

\end{answer}

\subsection{Returning a function}

Familiar mathematical functions take a real number and return a real
number. However, the concept of a function is more general. With
\texttt{julia} it is useful to write functions that take functions as
arguments, and return a derived function as an output. For the return
value an \emph{anonymous function} is typically used. These are
specified as \texttt{arg -\textgreater{} rule}, where the arg is a
variable, like \texttt{x}, and the rule can depend on the value of
\texttt{x}.

\begin{itemize}
\itemsep1pt\parskip0pt\parsep0pt
\item
  Describe what the following function does to the argument $f$, when
  $f$ is a function. (There isn't anything to do but recognize that
  \texttt{n} takes a function as input and returns a function as output,
  this question is how is \texttt{n(f)} related to \texttt{f}.)
\end{itemize}



\begin{verbatim}
n(f::Function) = x -> -f(x)
\end{verbatim}
\begin{answer}
type: longtext
reminder: Describe \verb+n(f::Function) = x -> -f(x)+
answer_text: The defines g(x) = -f(x) 
rows: 3
cols: 60
\end{answer}

\end{document}

